% !TEX root = ../dissertation.tex

\chapter{Mathematical Background}
This is some text
\section{Translational Dynamics}

\section{Rotational Dynamics}

\section{Coupled Dynamics}
\subsection{Dumbbell Spacecraft Equations of Motion}\label{sec:dumbbell}

The configuration space for rigid body motion is the semi-direct product, \(\SE = \R^3 \times \SO \), namely the special euclidean group.
The variations should be carefully constructed such that they respect the geometry of the configuration space.
By expressing the motion of the dumbbell directly on the special euclidean group, we avoid the issues inherent in using other kinematic representations which fail to preserve the geometric properties of the configuration space.
The kinematics of the dumbbell and asteroid are described in the inertial frame by
\begin{itemize}
    \item \( \vecbf{x} \in \R^3 \): the position of the center of mass of the dumbbell spacecraft represented in the inertial frame \( \vecbf{e}_i\)
    \item \( R \in \SO\): the rotation matrix which transforms vectors defined in the spacecraft fixed frame, \( \vecbf{b}_i \), to the inertial frame, \( \vecbf{e}_i \)
    \item \( \vecbf{\Omega} \in \R^3 \): the angular velocity of the spacecraft body fixed frame relative to the inertial frame and represented in the dumbbell body fixed frame \( \vecbf{b}_i \)
    \item \( R_A \in \SO \): the rotation matrix which transforms vectors defined in the asteroid fixed frame, \( \vecbf{f}_i \), to the inertial frame, \( \vecbf{e}_i \)
\end{itemize}
In this work, we assume that the asteroid is much more massive than the spacecraft and its motion is not affected by that of the spacecraft.
This assumption allows us to treat the motion of the vehicle independently from that of the asteroid, instead of treating the more complicated full-body problem. 

Using our kinematic variables we can define the kinetic and potential energy of the dumbbell as
\begin{align}\label{eq:kinetic_energy}
    T &= \frac{1}{2} m \norm{\dot{\vb{x}}}^2 + \frac{1}{2} \tr{S(\vb{\Omega}) J_d S\parenth{\vb{\Omega}}^T} , \\
    V( \vecbf{x}, R ) &=  - m_1 U \parenth{R_A^T \parenth{\vecbf{x} + R \vecbf{\rho}_1}} - m_2 U \parenth{R_A^T \parenth{\vecbf{x} + R \vecbf{\rho}_2}} ,
\end{align}
where the polyhedron potential is defined in~\cref{eq:potential}.
The position of each mass \(m_i\) of the dumbbell is defined in the dumbbell fixed frame by the vector \(\vb{\rho}_i\). 
The next step is to define the variations of the kinetic and potential energy to derive the equations of motion, which are given as
\begin{align} 
    \delta V &= -\sum_{i=1}^2  m_i \parenth{R_A \deriv{U}{\vb{z}_i} }^T \delta \vb{x} + m_i \hat{\vb{\eta}}\cdot \hat{\vb{\rho}_1} R^T R_A \deriv{U}{\vb{z}_i}, \\
    \delta T &= \parenth{m_1 + m_2} \dot{\vecbf{x}}^T \delta \dot{\vb{x}} + \frac{1}{2} \tr{- \dot{\hat{\vb{\eta}}} S(J \vb{\Omega}) + \hat{\vb{\eta}} S(\hat{\vb{\Omega}} J \vb{\Omega})}. 
\end{align}

Using the variations of the kinetic and potential energy we can derive the equations of motion of the dumbbell spacecraft about an asteroid using Hamilton's principle. 
Hamilton's principle then states that the variation of the action integral
\begin{align}
    \mathsf{G} = \int_{t_0}^{t_f} T(\dot{q}) - V(q) dt,
\end{align}
is stationary with fixed endpoints. 
Applying the calculus of variations and integration by parts results in the familiar Euler-Lagrange equations of motion.
Applying the Legendre transformation allows for the same dynamics to be expressed in an equivalent form as Hamilton's equations~\cite{lanczos1970}.
The equations of motion of a dumbbell spacecraft influenced by a polyhedron potential model are given as
\begin{align}
    \dot{\vb{x}} &= \vb{v}, \label{eq:position_kinematics}\\
    \parenth{m_1 + m_2} \dot{\vecbf{v}} &= m_1 R_A \deriv{U}{\vecbf{z}_1} + m_2 R_A \deriv{U}{\vecbf{z}_2} + \vecbf{u}_f, \label{eq:translational_dynamics}\\
    \dot{R} &= R S(\vb{\Omega}) , \label{eq:attitude_kinematics}\\
    J \dot{\vecbf{\Omega}} + \vecbf{\Omega} \times J \vecbf{\Omega} &= \vecbf{M}_1 + \vecbf{M}_2 + \vecbf{u}_m. \label{eq:attitude_dynamics}
\end{align}
The vectors \( \vecbf{z}_1 \) and \( \vecbf{z}_2\) define the position of the dumbbell masses as represented in the asteroid fixed frame and are defined as
\begin{align}
    \vecbf{z}_1 &= R_A^T \parenth{\vecbf{x} + R \vecbf{\rho}_1} , \\
    \vecbf{z}_2 &= R_A^T \parenth{\vecbf{x} + R \vecbf{\rho}_2}, 
\end{align}
where \( \vb{\rho}_i \) defines the position of each mass in the spacecraft fixed body frame.
The gravitational moment on the dumbbell \( \vecbf{M}_i\) is defined as
\begin{align}
    \vecbf{M}_i = m_i \parenth{S(R_A^T \vb{\rho}_i) R^T \deriv{U}{\vb{z}_i}}.
\end{align}
The control inputs to the spacecraft are defined by \( \vb{u}_f, \vb{u}_m \) which define the control force represented in the inertial frame and the control moment represented in the spacecraft frame, respectively. 

\section{Small-Body Shape Modelling}

Asteroids can have a wide variety of shapes, and most are vastly different than that of a sphere or ellipsoid.
\Cref{fig:irregular_asteroids} shows two examples of small solar system bodies that have highly irregular shapes.
As a result of these highly variable shapes, an adaptable format is required to represent the wide variety of surface shapes and features.
\begin{figure}[h]
    \centering
    \subcaptionbox{Asteroid Itokawa\label{fig:itokawa}}{\includegraphics[height=0.3\textheight,width=0.5\textwidth, keepaspectratio]{figures/mathematical_background/eso1405b.jpg}}~
    \subcaptionbox{Comet 67/Churyumov-Gerasimenko\label{fig:67p}}{\includegraphics[height=0.3\textheight,width=0.5\textwidth,keepaspectratio]{figures/mathematical_background/comet67pcgin.jpg}}
    \caption{Examples of the non-ellipsoidal shapes of small solar system bodies. Both asteroids and comets will tend to have vastly irregular shapes due to their low mass and violent histories of impacts and collisions~\label{fig:irregular_asteroids}}
\end{figure}
The scientific community describes asteroid shape using the facet-vertex model.
This approach is an efficient representation of the more general notion of a polyhedron from geometry.
Here we define the notion of the polyhedron and some specifics of the format used in the astrodynamics community.

A polyhedron is a generalization of a two-dimensional polygon to three-dimensions~\cite{orourke1998}.
It is the region of space with a boundary defined by a surface of a finite number of polygonal faces.
The surface of the polyhedron is composed of three types of primitive objects: zero-dimensional points called vertices, one-dimensional segments called edges, and two-dimensional polygons called faces or facets.
Furthemore, without any loss of generality we assume each face is a convex polygon since any nonconvex face can be divided into smaller convex faces.
A valid polyhedron in the context of asteroid shape models must satisfy several constraints.
These constraints define the relationship between each of the types of primitives which make up the polyhedron surface.
The primitives must intersect ``properly'', the local and global topology must be ``proper''.
For asteroid shape model we further assume that each face is a triangular polygon. 
Again, this does not limit generality as any polygon can be divided into a series of planar triangles.

The intersection of each face must be one of the following:
\begin{itemize}
    \item the faces are disjoint and do not intersect, or
    \item the faces meet at a single vertex, or
    \item the faces share two vertices and a common edge.
\end{itemize}
This constraint automatically ensures that all edgeds and vertices intersect properly.
Improper intersection would penetrating faces and faces that intersect improperly. 
Such as an edge not extending across an entire face.

The second constraint is related to the local topology around each point of the polyhedron.
In order to be locally proper, the neighborhood about any point on the surface of the polyhedron should be homeomorphic to a two-dimensional disk.
A neighborhood about any point on the surface is defined as an arbitrarily small subset or region of the surface which surrounds the point.
Every point on the surface should have a neighborhood which is topologically equivalent to a two dimensional disk.
The notion of equivalency is mathematically captured using the property of homeomorphism.
A homemorphism between two regions is a continous stretching or bending, without tearing or cutting, from one shape to another.
For example, it is possible to turn a circle into a square by a continuous stretching and bending of the shape.
However, it is not possible to transform a sphere into a torus, as this would require a hole to be created in the surface of the sphere.
The neighborhood about any point on the surface of the polyhedron should be equivalent to that of a two-dimensional disk.
A surface where this true for all points is called a \textit{two-manifold}, of a which the surface of a polyhedron is a subset.

The final constraint is related to the global structure of the surface in contrast to the local neighborhood of a point.
The surface must be connected, closed, and bounded.
In this sense, a connected surface is one where it is possible to travel from one point to any other point of the surface without leaving the surface. 
As a result, this will rule out any shapes with non-connected faces, such as a cube with a hollow interior/surface.
For example, on the outer surface of the cube it is not possible to reach a point on the interior surface. 
Combined with an assumption that there are a finite number of faces automatically ensures a closed and bounded surface. 
Note that these conditions do not in general rule out the possibility of holes passing completely through the object.
For example, a torus, or a donute shape, is also considered a polyhedron.
The key difference between a hole and a cavity is that there are no disconnected surfaces and as a result a polyhedron can have any number of such holes. 
In practice, we tend to limit our analysis to polyhedron with no holes, or with a genus of zero.

\subsubsection{Wavefront OBJ files}
The OBJ format is a geometry definition file format used for a variety of computer modeling applications, and is regularly used by the asteroid community~\cite{neese2004}.
The basic format of the file is an ASCII file where the first \( N_v\) lines begin with \texttt{v} and define the three components of a vertex in the body fixed reference frame.
The following \( N_f\) lines begin with \texttt{f} and define the three indices of the vertices that make up the face.
The numbering of the vertices is implicitly defined by the order listed in the file, i.e. the vertices are defined from \( 1 \) to \( N_v\).
There are two main assumptions used by the asteroid community.
First, each face is triangular and second, the vertices are numbered in a counterclockwise fashion about each face.
This allows the outward facing normal to each face to be uniquely defined without any additional data.

\section{Gravitational Models around Small-Bodies}

\subsection{Polyhedron Potential Model}\label{sec:polyhedron_potential}

% most use a spherical harmonic model or a ellipsoid model but we use a polyhedron model
An accurate gravitational potential model is necessary for the operation of spacecraft about asteroids.
Additionally, a detailed shape model of the asteroid is needed for trajectories passing close to the body.
The classic approach is to expand the gravitational potential into a harmonic series and compute the series coefficients.
However, the harmonic expansion is always an approximation as a result of the infinite order series used in the representation.
Additionally, the harmonic model used outside of the circumscribing sphere is not guaranteed to converge inside the sphere, which makes it unsuitable for trajectories near the surface.

We represent the gravitational potential of the asteroid using a polyhedron gravitation model.
This model is composed of a polyhedron, which is a three-dimensional solid body, that is defined by a series of vectors in the body-fixed frame.
The vectors define vertices in the body-fixed frame as well as planar faces which compose the surface of the asteroid.
We assume that each face is a triangle composed of three vertices and three edges.
As a result, only two faces meet at each edge while three faces meet at each vertex.
Only the body-fixed vectors, and their associated topology, is required to define the exterior gravitational model.
References~\cite{werner1994} and~\cite{werner1996} give a detailed derivation of the polyhedron model.
Here, we summarize the key developments and equations required for implementation.

Consider three vectors \( \vecbf{v}_1, \vecbf{v}_2, \vecbf{v}_3 \in \R^{3 \times 1} \), assumed to be ordered in a counterclockwise direction about an outward facing normal vector, which define a face.
It is easy to define the three edges of each face as
\begin{align}\label{eq:edges}
    \vecbf{e}_{i+1,i} = \vecbf{v}_{i+1} - \vecbf{v}_i \in \R^{3 \times 1 },
\end{align}
where the index \( i \in \parenth{1,2,3} \) is used to permute all edges of each face.
Since each edge is a member of two faces, there exist two edges which are defined in opposite directions between the same vertices.
We can also define the outward normal vector to face \( f\)  as
\begin{align}\label{eq:face_normal}
    \hat{\vecbf{n}}_f &= \parenth{\vecbf{v}_{2} - \vecbf{v}_1} \times \parenth{\vecbf{v}_{3} - \vecbf{v}_2} \in \R^{3 \times 1},
\end{align}
and the outward facing normal vector to each edge as
\begin{align}\label{eq:edge_normal}
    \hat{\vecbf{n}}_{i+1,i}^f &= \parenth{\vecbf{v}_{i+1} - \vecbf{v}_i} \times \hat{\vecbf{n}}_f \in \R^{3 \times 1}.
\end{align}
For each face we define the face dyad \( \vecbf{F}_f \) as
\begin{align}\label{eq:face_dyad}
    \vecbf{F}_f &= \hat{\vecbf{n}}_f \hat{\vecbf{n}}_f \in \R^{3 \times 3}.
\end{align}
Each edge is a member of two faces and has an outward pointing edge normal vector, given in~\cref{eq:edge_normal}, perpendicular to both the edge and the face normal.
For the edge connecting the vectors \( \vecbf{v}_1 \) and \( \vecbf{v}_2 \), which are shared between the faces \(A\) and \( B\), the per edge dyad is given by
\begin{align}\label{eq:edge_dyad}
    \vecbf{E}_{12} = \hat{\vecbf{n}}_A \hat{\vecbf{n}}_{12}^A + \hat{\vecbf{n}}_B \hat{\vecbf{n}}_{21}^B \in \R^{3 \times 3}.
\end{align}
The edge dyad \( \vecbf{E}_e  \), is defined for each edge and is a function of the two adjacent faces meeting at that edge.
The face dyad \( \vecbf{F}_f \), is defined for each face and is a function of the face normal vectors.

Let \( \vecbf{r}_i \in \R^{3 \times 1} \) be the vector from the spacecraft to the vertex \( \vecbf{v}_i \) and it's length is given by \( r_i = \norm{\vecbf{r}_i} \in \R^{1} \).
The per-edge factor \( L_e \in \R^{1}\), for the edge connecting vertices \( \vecbf{v}_i \) and \( \vecbf{v}_j \), with a constant length \( e_{ij} = \norm{\vecbf{e}_{ij}} \in \R^1\) is
\begin{align}\label{eq:edge_factor}
    L_e &= \ln \frac{r_i + r_j + e_{ij}}{r_i + r_j - e_{ij}}.
\end{align}
For the face defined by the vertices \( \vecbf{v}_i, \vecbf{v}_j, \vecbf{v}_k \) the per-face factor \( \omega_f \in \R^{1} \) is
\begin{align}\label{eq:face_factor}
    \omega_f &= 2 \arctan \frac{\vecbf{r}_i \cdot \vecbf{r}_j \times \vecbf{r}_k}{r_i r_j r_k + r_i \parenth{\vecbf{r}_j \cdot \vecbf{r}_k} + r_j \parenth{\vecbf{r}_k \cdot \vecbf{r}_i} + r_k \parenth{\vecbf{r}_i \cdot \vecbf{r}_j}}.
\end{align}
The gravitational potential due to a constant density polyhedron is given as
\begin{align}\label{eq:potential}
    U(\vecbf{r}) &= \frac{1}{2} G \sigma \sum_{e \in \text{edges}} \vecbf{r}_e \cdot \vecbf{E}_e \cdot \vecbf{r}_e \cdot L_e - \frac{1}{2}G \sigma \sum_{f \in \text{faces}} \vecbf{r}_f \cdot \vecbf{F}_f \cdot \vecbf{r}_f \cdot \omega_f \in \R^1,
\end{align}
where \( \vecbf{r}_e\) and \(\vecbf{r}_f \) are the vectors from the spacecraft to any point on the respective edge or face, \( G\) is the universal gravitational constant, and \( \sigma \) is the constant density of the asteroid.
Furthermore we can use these definitions to define the attraction, gravity gradient matrix, and Laplacian as
\begin{align}
    \nabla U ( \vecbf{r} ) &= -G \sigma \sum_{e \in \text{edges}} \vecbf{E}_e \cdot \vecbf{r}_e \cdot L_e + G \sigma \sum_{f \in \text{faces}} \vecbf{F}_f \cdot \vecbf{r}_f \cdot \omega_f \in \R^{3 \times 1} , \label{eq:attraction}\\
    \nabla \nabla U ( \vecbf{r} ) &= G \sigma \sum_{e \in \text{edges}} \vecbf{E}_e  \cdot L_e - G \sigma \sum_{f \in \text{faces}} \vecbf{F}_f \cdot \omega_f \in \R^{3 \times 3}, \label{eq:gradient_matrix}\\
    \nabla^2 U &= -G \sigma \sum_{f \in \text{faces}}  \omega_f \in \R^1.\label{eq:laplacian}
\end{align}

One interesting thing to note is that both~\cref{eq:face_dyad,eq:edge_dyad} can be precomputed without knowledge of the position of the satellite.
They are both solely functions of the vertices and edges of the polyhedral shape model and are computed once and stored.
Once a position vector \( \vecbf{r} \) is defined, the scalars given in~\cref{eq:edge_factor,eq:face_factor} can be computed for each face and edge.
Finally,~\cref{eq:potential} is used to compute the gravitational potential on the spacecraft.
The Laplacian, defined in~\cref{eq:laplacian}, gives a simple method to determine if the spacecraft has collided with the body~\cite{werner1996}. 


