\documentclass[11pt, reqno]{article}   	% use "amsart" instead of "article" for AMSLaTeX format
\usepackage{my_packages}

\title{Vector Norm}
\author{Shankar Kulumani}
\date{12 January 2016}

\newcommand{\F}{\mathbb{F}}
\newcommand{\V}{\mathsf{V}}

\begin{document}
\maketitle
\section{Vector Space}

\begin{definition}
	A field, \( \F \), is a set of elements, called \emph{scalars}, and two operations denoted by \( +\) and \( \cdot\).
	The following properties must hold
	\begin{enumerate}
		\item \emph{Closure}: For all \( a, b \in \F \) both \( a + b \in \F \) and \( a \cdot b \in \F \).
			More formally this means that \( + \) and \( \cdot \) are binary operations on \( \F \).
		\item \emph{Commutativity}: For all \( a, b \in \F \) the following hold: \( a + b = b+a \) and \( a \cdot b = b \cdot a\).
		\item \emph{Associativity}: For all \( a, b, c \in \F \) the following hold: \( a + \parenth{b + c} = \parenth{a + b} + c\) and \( a \cdot \parenth{b \cdot c} = \parenth{a \cdot b} \cdot c\)
		\item \emph{Distributivity}: For all \( a, b, c \in \F\) the following holds: \( a \cdot \parenth{b+c} = \parenth{a\cdot b} + \parenth{a \cdot c} \).
		\item \emph{Zero Element}: There exists an element \( 0 \in \F \) such that for all \( a \in \F, a + 0 = a \).
		\item \emph{Identity Element}: There exists an element \( 1 \in \F \) such that for all \( a \in \F, 1 \cdot a = a \).
		\item \emph{Inverse}: For all \( a \in \F \) there exists \( b \in \F \) such that \( a + b = 0 \). 
			Similiarly, for all \( a \neq 0\) there exists \( b \in \F \) such that \( a \cdot b = 1 \)
	\end{enumerate}
\end{definition}

A vector space over a field \( \F \) is a set \( \V \) together with the properties above. 
Elements of \( \V \) are commonly called vectors while elements of \( \F \) are commonly called scalars.
The operations of vector addition and scalar multiplication hold such that: \( \vecbf{v} + \vecbf{w} \in \V \) and \( a \vecbf{v} \in \V \) for all \( a \in \F \)

\section{Inner Product and norm}
\begin{definition}
    A vector norm is a function \( f(\vecbf{x}) : \R^n \to \R \) with specific properties.
    If \( \vecbf{x} \in \R^n \), we denote its norm by \( \norm{\vecbf{x}} \).
    The vector norm posses the following properties
    \begin{enumerate}
    	\item \emph{Positivity}: \( \norm{\vecbf{x}} \geq 0 \) for all \( \vecbf{x} \in \R^n \) and \( \norm{\vecbf{x}} = 0 \) if and only if \( \vecbf{x} = \mathbf{0} \)
    	\item \emph{Homegeneity}: \( \norm{\alpha \vecbf{x}} = | \alpha | \norm{x} \) for all \( \alpha \in \R, \vecbf{x} \in \R^n \)
    	\item \emph{Triangle Inequality}: \( \norm{ \vecbf{x} + \vecbf{y}} \leq \norm{\vecbf{x}} + \norm{\vecbf{y}} \) for all \( \vecbf{x},\vecbf{y} \in \R^n \)
    \end{enumerate}
\end{definition}


\begin{definition}
	An \emph{inner product} on the vector space \( \V \) over the field \( \F \) is map which assigns to each pair of vectors \( a, b \in \V\) a value \( \pair{a,b} \in F \) or equivalently,
	\[
		\pair{\cdot,\cdot} : \V \times \V \to \F .
	\]
	This inner products satisfies three axioms for vectors \( a,b,c \in \V \) and scalars \( c_1,c_2 \in \F\):
	\begin{enumerate}
		\item \emph{Symmetry}: \( \pair{a,b} = \bar{\pair{b,a}} \)
		\item \emph{Bilinearity}: \(\pair{c_1 a +c_2 b,c} = c_1 \pair{a,c} + c_s \pair{b,c} \)
		\item \emph{Positive-Definiteness}: \( \pair{a,a} \geq 0 \) and \( \pair{a,a} = \) if and only if \( a = 0 \)
	\end{enumerate}
\end{definition}

An \emph{inner product space} is a pair \( \parenth{\V, \pair{\cdot,\cdot}} \) where \( \pair{\cdot,\cdot} \) is an inner product on \( \V \).
A \emph{normed vector space} is a pair \( \parenth{\V, \norm{\cdot}} \), where \( \norm{\cdot} \) is a norm on \(\V\).
An inner product space is an example of a normed vector space with the norm given by \( \norm{\vecbf{x}} = \sqrt{\pair{\vecbf{x},\vecbf{x}}} \).


\subsection{Norm Examples}

\section{Matrix Norm}

\section{Useful Properties}
\begin{definition}
	\emph{Cauchy-Schwarz inequality}: Every inner product satisfies the following
	\[
		\left| \pair{\vecbf{a},\vecbf{b} }\right| \leq \norm{\vecbf{a}} \norm{\vecbf{b}} \, \forall \, \vecbf{a},\vecbf{b} \in \V
	\]
\end{definition}


\bibliographystyle{abbrv}
\bibliography{library.bib}


\end{document}  