% !TEX root = ../../defense.tex

\section{Dynamic Model}

\begin{frame}{Reference Frames}

    \begin{columns}
        \begin{column}{0.5\textwidth}
            \begin{enumerate}
                \item \( \vc{e}_i \in \R^3 \) : inertial reference frame fixed in space
                \item \( \vc{f}_i \in \R^3 \) : small body fixed frame aligned with principle axes of the body
                \item \( \vc{b}_i \in \R^3 \) : spacecraft fixed frame with \( b_1 \) aligned with the symmetry axis
            \end{enumerate}
            
            \begin{block}{}
                Spacecraft state is defined on the Special Euclidean group

                \[ \parenth{R, \vc{x}} \in \SE \]
            \end{block}
        \end{column}
        \begin{column}{0.5\textwidth}
            \begin{scaletikzpicturetowidth}{2\columnwidth}
                \resizebox{\columnwidth}{!}{%
                \input{tikz/ref_frames.tikz}
            }
            \end{scaletikzpicturetowidth}
        \end{column}
    \end{columns}
\end{frame}

\begin{frame}{Spacecraft Model}
    \begin{itemize}
        \item Spacecraft modeled as a rigid dumbbell
        \item Two masses \( m_1, m_2 \in \R^1 \) connected by a rigid massless link \( l \in \R^1\)
        \item Effectively captures the mass distribution of a spacecraft
    \end{itemize}

    \resizebox{\textwidth}{!}{%
        \input{tikz/dumbbell.tikz}
    }
\end{frame}

\begin{frame}{Equations of Motion}
    Hamilton's principle used to derive equations of motion in four forms:
    \begin{enumerate}
        \item Inertial vs. Relative 
        \item Lagrangian vs. Hamiltonian
    \end{enumerate}
    True motion of the system is a stationary point of the action integral
    \begin{align*}
        \mathcal{G} &= \int_{t_0}^{t_f} T(\dot q) - V(q) dt\\
        \delta \mathcal{G} &= \int_{t_0}^{t_f} \delta T - \delta V dt = 0
    \end{align*}
\end{frame}

\begin{frame}{Inertial EOMs in Hamiltonian Form}
\begin{align}
    \dot \ipos &= \frac{\ilinmom}{m_1 + m_2}, \\
    \dot \ilinmom &= - m_1 \aatt \deriv{U}{\apos_1} - m_2 \aatt \deriv{U}{\apos_2}, \\
    \dot \iatt &= R \parenth{J^{-1} \iangmom}^\wedge, \\
    \dot\iangmom  + \hat{\iangvel} \iangmom &= m_1 \hat{\spos}_1 \iatt^T \aatt \deriv{U}{\apos_1} + m_2 \hat{\spos}_2 \iatt^T \aatt \deriv{U}{\apos_2}. 
\end{align}

\end{frame}

\begin{frame}{Relative Equations of Motion}

\end{frame}
