% !TEX root = ../../defense.tex

\section{System Model}
\subsection{EOMs}

\begin{frame}{Reference Frames}

    \begin{columns}
        \begin{column}{0.5\textwidth}
            \begin{enumerate}
                \item \( \vc{e}_i \in \R^3 \) : inertial reference frame fixed in space
                \item \( \vc{f}_i \in \R^3 \) : small body fixed frame aligned with principle axes of the body
                \item \( \vc{b}_i \in \R^3 \) : spacecraft fixed frame with \( b_1 \) aligned with the symmetry axis
            \end{enumerate}
            
            \begin{block}{}
                Spacecraft state is defined on the Special Euclidean group

                \[ \parenth{R, \vc{x}} \in \SE \]
            \end{block}
        \end{column}
        \begin{column}{0.5\textwidth}
            \begin{scaletikzpicturetowidth}{2\columnwidth}
                \resizebox{\columnwidth}{!}{%
                \input{tikz/ref_frames.tikz}
            }
            \end{scaletikzpicturetowidth}
        \end{column}
    \end{columns}
\end{frame}

\begin{frame}{Spacecraft Model}
    \begin{itemize}
        \item Spacecraft modeled as a rigid dumbbell
        \item Two masses \( m_1, m_2 \in \R^1 \) connected by a rigid massless link \( l \in \R^1\)
        \item Effectively captures the mass distribution of a spacecraft
    \end{itemize}

    \resizebox{\textwidth}{!}{%
        \input{tikz/dumbbell.tikz}
    }
\end{frame}

\begin{frame}{Equations of Motion}
    Hamilton's principle used to derive equations of motion in four forms:
    \begin{enumerate}
        \item Inertial vs. Relative 
        \item Lagrangian vs. Hamiltonian
    \end{enumerate}
    True motion of the system is a stationary point of the action integral
    \begin{align*}
        \mathcal{G} &= \int_{t_0}^{t_f} T(\dot q) - V(q) dt\\
        \delta \mathcal{G} &= \int_{t_0}^{t_f} \delta T - \delta V dt = 0
    \end{align*}
\end{frame}

\begin{frame}{Inertial EOMs in Lagrangian Form}
    The kinetic and potential energy of the dumbbell in the inertial frame
    \begin{align*}
        T\parenth{\dot{\ipos}, \iangvel} &= \frac{1}{2} m \norm{\dot{\vc{x}}}^2 + \frac{1}{2} \tr{\vh{\Omega} J_d \vh{\Omega}^T} , \\
        V( \ipos, R ) &=  - m_1 U \parenth{R_A^T \parenth{\vc{x} + R \vc{\rho}_1}} - m_2 U \parenth{R_A^T \parenth{\vc{x} + R \vc{\rho}_2}}. 
    \end{align*}
    The stationary point of the action integral gives:
    \begin{align*}
        \dot{\ipos} &= \ivel, \\
        \parenth{m_1 + m_2} \dot{\ivel} &= - \sum_{i=0}^2 m_i \aatt \deriv{U}{\apos_i} + u_f, \\
        \dot{\iatt} &= \iatt \iangvel, \\
        J \dot{\iangvel} + \hat{\iangvel} J \iangvel &= \sum_{i=0}^2 m_i \hat{\spos}_i \iatt^T \aatt \deriv{U}{\apos_i} + u_m. 
    \end{align*}
\end{frame}

\begin{frame}{Relative Equations of Motion}

\begin{itemize}
    \item Gravitational potential is a function of relative position
    \item Can derive EOMs in the rotating small body fixed frame
    \item Kinematics defined relative to the small body
\end{itemize}

\begin{align*}
    \dot{x}_r &= \rvel - \hat{\Omega}_A \rpos, \\
    \parenth{m_1 + m_2} \dot{v}_r  &= -\parenth{m_1 + m_2} \hat{\Omega}_A \rvel + m_1 \deriv{U}{z_1} + m_2 \deriv{U}{z_2}, \\
    \dot{R}_r &= \hat{\Omega}_r \ratt  - \hat{\Omega}_A \ratt , \\
    \dot{\parenth{\Jr  \rangvel}} + \hat{\Omega}_A \parenth{\Jr  \rangvel} &= m_1 \parenth{\ratt  \rho_1}^\wedge \deriv{U}{z_1} + m_2 \parenth{\ratt  \rho_2}^\wedge \deriv{U}{z_2}.
\end{align*}
\end{frame}

\subsection{Gravity}

\begin{frame}{Gravitational Models}
    \begin{center}
        \tikzsetnextfilename{newton_gravity}
\begin{tikzpicture}
    \coordinate (m1) at (0, 0);
    \coordinate (m2) at (8, 0);

    \shade[ball color=blue, opacity=0.7] (m1) circle (1.5);
    \shade[ball color=red, opacity=0.7] (m2) circle (1);

    \node[ above] at (m1) {$m_1$};
    \node[above] at (m2) {$m_2$};

    \draw[-Latex] (m1) -- ++(3, 0) node[below ] {$r$}; 
    \draw[-Latex] (m2) -- ++(-2, 0) node[above] {$F_g$};
\end{tikzpicture}

    \end{center}

    \begin{itemize}
        \item Newton's law of gravity only applies to spherical bodies
        \item Variety of methods to represent irregular bodies
            \begin{itemize}
                \item Spherical Harmonic - not valid when close to body
                \item Constant Density Ellipsoid - only valid for ellipsoids
                \item \Emph{Polyhedron Potential} - function of shape model
            \end{itemize}
    \end{itemize}
\end{frame}

\begin{frame}{Polyhedron Gravitation Model}

\begin{columns}
\begin{column}{0.5\textwidth}
\begin{itemize}
    \item Function of shape alone
    \item Globally valid and closed form
    \item Assumes a constant density
    \item Accuracy of the potential depends on the shape estimate
\end{itemize}
\end{column}
\begin{column}{0.5\textwidth}
    \resizebox{\columnwidth}{!}{%
    \input{tikz/polyhedron_faces.tikz}
}
\end{column}
\end{columns}
\visible<2>{
\begin{align*}
    U(\vc{r}) =& \frac{1}{2} G \sigma \sum_{e \in \text{edges}} \vc{r}_e \cdot \vc{E}_e \cdot \vc{r}_e \cdot L_e - \frac{1}{2}G \sigma \sum_{f \in \text{faces}} \vc{r}_f \cdot \vc{F}_f \cdot \vc{r}_f \cdot \omega_f 
\end{align*}
}
\end{frame}
