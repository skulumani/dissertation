% !TEX root = ../../dissertation.tex

\chapter{Derivation of Inertial Equations of Motion}\label{proof:inertial_dumbbell_eoms}
The Lagrangian function \( L \in \R \) is a scalar function of the generalized coordinates and their velocities..
The Lagrangian is defined as the total system kinetic energy minus the total system potential energy:
\begin{align*}
    L(q, \dot{q} ) = T(q, \dot{q}) - U (q) ,
\end{align*}
where \( T(q, \dot{q} \) is the kinetic energy of the system while \( U(q) \) is the potential energy of the system.

The derivation is simplified to the case of two spherical masses, but can be extended to any arbitrary configuration of masses.
The kinetic and potential energy of the dumbbell model is dupblicated from~\cref{eq:dumbbell_kinetic_and_potential_energy} 
\begin{align*}
    T &= \frac{1}{2} m \norm{\dot{\vb{x}}}^2 + \frac{1}{2} \tr{\vh{\Omega} J_d \vh{\Omega}^T} , \\
    V( \vecbf{x}, R ) &=  - m_1 U \parenth{R_A^T \parenth{\vecbf{x} + R \vecbf{\rho}_1}} - m_2 U \parenth{R_A^T \parenth{\vecbf{x} + R \vecbf{\rho}_2}} .
\end{align*}
The configuration space of the motion is \( \SE \), and as a result the variations must be correctly defined to respect the geometry of the configuration space.
Consider the variations of \( \vc{x}, \dot{\vc{x}} \) as
\begin{align}
    \vc{x}^\epsilon &= \vc{x} + \epsilon \delta \vc{x} + \mathcal{O}(\epsilon^2), \label{eq:inertial_pos_variation} \\
    \dot{\vc{x}}^\epsilon &= \dot{\vc{x}} \delta \dot{\vc{x}} + \mathcal{O}(\epsilon^2), \label{eq:inertial_vel_variation}
\end{align}
where \( \delta \vc{x}, \delta \dot{\vc{x}} \in \R^3 \) are the infinitesimal variations of the position and velocity which vanish at the endpoints.
In other words,
\begin{align*}
    \delta \vc{x} (t_0) &= \delta \vc{x} (t_f ) = 0, \\
    \delta \dot{\vc{x}} (t_0) &= \delta \dot{\vc{x}} (t_f) = 0 .
\end{align*}
The variation of \( R \) is expressed in terms of the Lie algebra \( \vc{\eta} \in \so \) and the exponential map~\cite{hall2015} as
\begin{align}\label{eq:inertial_att_variation}
    \delta R =  \left.\diff{}{\epsilon}\right|_{\epsilon=0} \exp \epsilon \vc{\eta} = R \vc{\eta}
\end{align}
where the variation \( \vc{\eta} \) vanishes at \( t_0, t_f\).
The variation of \( \vc{\Omega} \) is computed from the kinematics and~\cref{eq:inertial_att_variation} as
\begin{align}
    \parenth{\delta \vc{\Omega}}^\wedge  &= \left. \diff{}{\epsilon}\right|_{\epsilon=0} {R^\epsilon}^T \dot{R}^\epsilon, \nonumber \\
                                         &= \delta R^T \dot{R} + R^T \delta \dot{R}, \nonumber \\
                                         &= - \vc{\eta} \vh{\Omega} + \vh{\Omega} + \dot{\vc{\eta}}\label{eq:inertial_ang_vel_variation}
\end{align}

\section{Variation of the kinetic energy}\label{sec:inertial_kinetic_energy_variation}
The variation of the kinetic energy is defined as
\begin{align}
    \delta T &= \left. \diff{}{\epsilon}\right|_{\epsilon=0} T(\dot{\vc{x}} + \epsilon \delta \dot{\vc{x}}, \vc{\Omega} + \epsilon \delta \vc{\Omega}), \nonumber\\
             &=   \parenth{m_1 + m_2} \dot{\vecbf{x}}^T \delta \dot{\vecbf{x}} + \frac{1}{2} \tr{ (\delta \vc{\Omega} )^\wedge J_d (\delta \vc{\Omega} )^T + (\delta \vc{\Omega})J_d (\delta \vc{\Omega})^T}\label{eq:kinetic_variation}.
\end{align}
Now,~\cref{eq:inertial_ang_vel_variation} is substituted into~\cref{eq:kinetic_variation} to obtain
\begin{align*}
    \delta T &=  \parenth{m_1 + m_2} \dot{\vecbf{x}}^T \delta \dot{\vb{x}} + \frac{1}{2} \tr{\parenth{-\eta S(\vb{\Omega}) + S(\vb{\Omega}) \eta + \dot{\eta}} J_d S(\vb{\Omega})^T + S(\vb{\Omega}) J_d \parenth{-\eta S(\vb{\Omega}) + S(\vb{\Omega}) \eta + \dot{\eta}}^T} .
\end{align*}
Applying the transpose and multiplying out terms gives
\begin{align*}
    \delta T &=  \parenth{m_1 + m_2} \dot{\vecbf{x}}^T \delta \dot{\vb{x}} \\
    &+ \frac{1}{2} \tr{ -\eta S(\vb{\Omega}) J_d S(\vb{\Omega})^T + S(\vb{\Omega})\eta J_d S(\vb{\Omega})^T + \dot{\eta}J_d S(\vb{\Omega})^T + S(\vb{\Omega}) J_d \parenth{-S(\vb{\Omega}) \eta + \eta S(\vb{\Omega}) - \dot{\eta}}} , \\
    &=  \parenth{m_1 + m_2} \dot{\vecbf{x}}^T \delta \dot{\vb{x}} \\
    &+ \frac{1}{2} \tr{-\eta S(\vb{\Omega}) J_d S(\vb{\Omega})^T + S(\vb{\Omega})\eta J_d S(\vb{\Omega})^T + \dot{\eta}J_d S(\vb{\Omega})^T -S(\vb{\Omega}) J_d S(\vb{\Omega}) \eta + S(\vb{\Omega}) J_d \eta S(\vb{\Omega}) - S(\vb{\Omega}) J_d \dot{\eta} } .
\end{align*}
Using the trace property, \( \tr{AB} = \tr{BA} = \tr{A^T B^T} = \tr{B^T A^T} \), lets us redefine several terms
\begin{align*}
    \tr{-S(\vb{\Omega}) J_d \dot{\eta}} &= \tr{- \dot{\eta} S(\vb{\Omega}) J_d} , \\
    \tr{S(\vb{\Omega}) J_d \eta S(\vb{\Omega})} &= \tr{\eta S(\vb{\Omega}) S(\vb{\Omega}) J_d} , \\
    \tr{-\eta S(\vb{\Omega}) J_d S(\vb{\Omega})^T} &= \tr{\eta S(\vb{\Omega}) J_d S(\vb{\Omega})} .
\end{align*}
These transformations allows us to combine several similar terms as
\begin{align*}
    \delta T &= \parenth{m_1 + m_2} \dot{\vecbf{x}}^T \delta \dot{\vb{x}} \\
    & + \frac{1}{2} \tr{-\dot{\eta} \parenth{J_d S(\vb{\Omega}) + S(\vb{\Omega}) J_d}} + \frac{1}{2} \tr{\eta S(\vb{\Omega}) \parenth{J_d S(\vb{\Omega}) + S(\vb{\Omega}) J_d}} + \frac{1}{2} \tr{ S(\vb{\Omega}) \eta J_d S(\vb{\Omega})^T - S(\vb{\Omega}) J_d S(\vb{\Omega}) \eta}
\end{align*}
The last trace term can be simplified using the properties of the trace and hat map as
\begin{align*}
    \frac{1}{2} \tr{ S(\vb{\Omega}) \eta J_d S(\vb{\Omega})^T - S(\vb{\Omega}) J_d S(\vb{\Omega}) \eta} &= \frac{1}{2} \tr{ - S(\vb{\Omega}) \eta J_d S(\vb{\Omega}) - S(\vb{\Omega}) J_d S(\vb{\Omega}) \eta} , \\
    &= \frac{1}{2} \tr{ -  \eta J_d S(\vb{\Omega}) S(\vb{\Omega}) - \eta S(\vb{\Omega}) J_d S(\vb{\Omega}) } \\
    &= -\frac{1}{2} \tr{\eta \parenth{J_d S(\vb{\Omega}) +S(\vb{\Omega}) J_d }S(\vb{\Omega})}
\end{align*}
Using this simplification gives
\begin{align*}
    \delta T &= \parenth{m_1 + m_2} \dot{\vecbf{x}}^T \delta \dot{\vb{x}} \\
    & + \frac{1}{2} \tr{-\dot{\eta} \parenth{J_d S(\vb{\Omega}) + S(\vb{\Omega}) J_d}} + \frac{1}{2} \tr{\eta S(\vb{\Omega}) \parenth{J_d S(\vb{\Omega}) + S(\vb{\Omega}) J_d}} - \frac{1}{2} \tr{ \eta \parenth{J_d S(\vb{\Omega}) +S(\vb{\Omega}) J_d }S(\vb{\Omega})} .
\end{align*}
From~\cref{eq:nonstandard_moi} this is redefined as
\begin{align}\label{eq:kinetic_energy_variation_final}
    \delta T &= \parenth{m_1 + m_2} \dot{\vecbf{x}}^T \delta \dot{\vb{x}} + \frac{1}{2} \tr{-\dot \eta S(J \vb{\Omega}) + \eta \parenth{ S(\vb{\Omega}) S(J \vb{\Omega}) + S(J \vb{\Omega}) S(\vb{\Omega})} } , \nonumber\\
    &=  \parenth{m_1 + m_2} \dot{\vecbf{x}}^T \delta \dot{\vb{x}} + \frac{1}{2} \tr{ - \dot{\eta} S(J \vb{\Omega}) + \eta S(\vb{\Omega} \times J \vb{\Omega})} , 
\end{align}
where we used the hat map property, \( S(\vb{x} \times \vb{y}) = S(\vb{x})S(\vb{y}) - S(\vb{y})S(\vb{x}) \), to arrive at the final form.


\section{Variation of the potential energy}\label{sec:inertial_potential_energy_variation}
With the appropriate variations defined on the configuration space when can define the variation of~\cref{eq:inertial_potential_energy}.
Applying the chain and product rule of differentiation results in
\begin{align}
    \delta V = - \sum_{i=1}^{2} m_i \deriv{U_i}{\vc{z_i}}^T \parenth{ R_A^T \delta \vc{x} + R_A^T \delta R \vc{\rho}_i}
\end{align}
where \( \vc{z}_i = R_A^T \parenth{\vc{x} + R \vc{\rho}_i} \) is the position of each mass in the asteroid fixed frame.
Applying the definition for the attitude variation, \(\delta R = R \eta \), the variation of the potential energy becomes
\begin{align*}
    \delta V = -\sum_{i=1}^2 \bracket{ m_i \parenth{R_A \deriv{U_i}{\vc{z}_i} }^T \delta \vc{x} + m_i \vc{\eta}^T \vh{\rho}_i R^T R_A \deriv{U_i}{\vb{z}_i}}
\end{align*}
which is equivalent to~\cref{eq:dumbbell_potential_energy_variation}.
