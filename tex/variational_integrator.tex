% description of variational integrator development
% !TEX root = ../dissertation.tex

\chapter{Variational Integrators for CRTBP}
\section{Variational Integrator}
A variational integrator is derived through discrete version of the action integral.
In the continuous time, for holonomic systems the action integral is defined as
\begin{align}\label{eq:action_integral}
	S = \int_{0}^T L\left( q, \dot{q}\right) \, dt
\end{align}
The equations of motion can be derived from the Lagrangian, \( L\left(q, \dot{q} \right) \), by taking the variation of~\cref{eq:action_integral}
\begin{align}\label{eq:el_eq}
	\delta S &= \int_{0}^T \deriv{L}{q} \delta q + \deriv{L}{\dot{q}} \delta \dot{q} \, dt \\
		&= \int_{0}^T \deriv{L}{q} \delta q - \frac{d}{dt} \left( \deriv{L}{\dot{q}}\right) \delta q \, dt - \left. \left[ \deriv{L}{\dot{q}} \delta q\right] \right|_0^T \\
	\delta S = 0 &= \deriv{L}{q} - \frac{d}{dt} \left( \deriv{L}{\dot{q}}	\right)
\end{align}
\Cref{eq:el_eq} is a vector second order differential equation in terms of the generalized position and velocities and results in \( n\) second order ODEs.

Hamilton's equations are derivable through the use of the Legendre transformation which is a mapping \( \left( q, \dot{q},t\right) \rightarrow \left(q, p, t \right) \) where \( p_i\) is the generalized momenta.
\begin{align}\label{eq:legendre_transform}
	p_i = \deriv{L}{\dot{q}}
\end{align}

The Legendre transformation is useful for variational equations
\begin{align*}
	df &= u dx + v dy \\
\end{align*}
where
\begin{align*}
	u &= \deriv{f}{x} & v &= \deriv{f}{y}
\end{align*}
To change from \( \left( x, y\right) \rightarrow \left(u,y \right) \)
\begin{align*}
	g &= f - u x\\
	dg &= df - u dx - x du\\
	dg &= v dy - x du \\
	x&= -\deriv{g}{u} & v&= \deriv{g}{y}
\end{align*}

In the continuous time case one can define the Hamiltonian
\begin{align}\label{eq:hamiltonian}
	H &= \sum_{i = 1}^N p_i \dot{q}_i - L \left( q_i,\dot{q}_i, t \right)
\end{align}
Applying~\cref{eq:legendre_transform} and taking the variation of~\cref{eq:hamiltonian} allows us to derive the equations of motion in Hamiltonian form
\begin{align}\label{eq:hamilton_eq}
	\dot{q}_i &= \deriv{H}{p_i} \\
	\dot{p}_i &= - \deriv{H}{q_i} + Q_i \\
	\deriv{L}{t} &= -\deriv{H}{t}
\end{align}
Again we are left with \( 2n \) first order differential equations that describe the system dynamics in terms of the generalized position and momenta \( q_i \) and \( p_i\), respectively.

\section{Discrete Lagrange Mechanics}
We approximate the action integral between \( q_0 \) and \( q_1 \) by quadrature
\begin{align}\label{eq:exact_discrete_lagrangian}
	L_D\left( q_0 , q_1 \right) = \int_{0}^{1} L \left( q , \dot{q} \right) \, dt
\end{align}
There are multiple possible methods one can use to approximate the integral in~\cref{eq:exact_discrete_lagrangian}.

\begin{align}\label{eq:quadrature_rules}
	\text{Rectangle Rule} & q &= q_0 & \dot{q} &= \frac{q_1 - q_0}{\Delta t}
\end{align}

% TODO : Complete the discussion on variational integrators
\begin{itemize}
	\item Different rules for Lagrangian
	\item Discrete version of the dynamics
	\item Discuss jacobians for use in the costate
	\item Show all the terms used in the analytical inverse operation
\end{itemize}
